\documentclass[a4paper]{article}
\usepackage[utf8]{inputenc}
\usepackage[russian]{babel}
\usepackage{booktabs}
\usepackage{geometry}
\usepackage{fancyhdr}
\usepackage{titlesec}
\usepackage{setspace}
\usepackage{amsmath, amssymb}
\usepackage{graphicx}
\usepackage{float}
\usepackage{listings}
\usepackage{tabularx}
\usepackage{caption}
\usepackage{color}
\usepackage{tcolorbox}
\usepackage{hyperref}
\hypersetup{hidelinks}

% ---- Поля страницы ----
\geometry{
    left=20mm,
    right=20mm,
    top=20mm,
    bottom=20mm
}

% ---- Межстрочный интервал ----
\onehalfspacing

% ---- Колонтитулы ----
\pagestyle{fancy}
\fancyhf{}
\fancyfoot[C]{\thepage}

% ---- Стиль заголовков ----
\titleformat{\section}{\bfseries\large}{\thesection}{1em}{}
\titleformat{\subsection}{\bfseries\normalsize}{\thesubsection}{1em}{}
\titleformat{\subsubsection}{\normalsize}{\thesubsubsection}{1em}{}

% ---- Настройки листингов ----
\lstset{
    basicstyle=\footnotesize\sffamily, 
    numbers=left,               
    numberstyle=\tiny,           
    stepnumber=1,                   
    numbersep=5pt,                
    backgroundcolor=\color{white},
    showspaces=false,            
    showstringspaces=false,
    showtabs=false,             
    frame=single,              
    tabsize=2,                 
    captionpos=t,              
    breaklines=true,           
    breakatwhitespace=false, 
    escapeinside={\%*}{*)},  
    inputencoding=utf8,
    extendedchars=true,
    literate={а}{{\char224}}1 {б}{{\char225}}1 {в}{{\char226}}1 {г}{{\char227}}1 {д}{{\char228}}1
             {е}{{\char229}}1 {ё}{{\char184}}1 {ж}{{\char230}}1 {з}{{\char231}}1 {и}{{\char232}}1
             {й}{{\char233}}1 {к}{{\char234}}1 {л}{{\char235}}1 {м}{{\char236}}1 {н}{{\char237}}1
             {о}{{\char238}}1 {п}{{\char239}}1 {р}{{\char240}}1 {с}{{\char241}}1 {т}{{\char242}}1
             {у}{{\char243}}1 {ф}{{\char244}}1 {х}{{\char245}}1 {ц}{{\char246}}1 {ч}{{\char247}}1
             {ш}{{\char248}}1 {щ}{{\char249}}1 {ъ}{{\char250}}1 {ы}{{\char251}}1 {ь}{{\char252}}1
             {э}{{\char253}}1 {ю}{{\char254}}1 {я}{{\char255}}1 {А}{{\char192}}1 {Б}{{\char193}}1
             {В}{{\char194}}1 {Г}{{\char195}}1 {Д}{{\char196}}1 {Е}{{\char197}}1 {Ё}{{\char168}}1
             {Ж}{{\char198}}1 {З}{{\char199}}1 {И}{{\char200}}1 {Й}{{\char201}}1 {К}{{\char202}}1
             {Л}{{\char203}}1 {М}{{\char204}}1 {Н}{{\char205}}1 {О}{{\char206}}1 {П}{{\char207}}1
             {Р}{{\char208}}1 {С}{{\char209}}1 {Т}{{\char210}}1 {У}{{\char211}}1 {Ф}{{\char212}}1
             {Х}{{\char213}}1 {Ц}{{\char214}}1 {Ч}{{\char215}}1 {Ш}{{\char216}}1 {Щ}{{\char217}}1
             {Ъ}{{\char218}}1 {Ы}{{\char219}}1 {Ь}{{\char220}}1 {Э}{{\char221}}1 {Ю}{{\char222}}1
             {Я}{{\char223}}1
}

% ---- Включение директории с изображениями ----
\graphicspath{{figures/}}

% ---- Метаданные (замените значения) ----
\newcommand{\university}{Санкт-Петербургский политехнический университет Петра Великого}
\newcommand{\faculty}{Институт компьютерных наук и кибербезопасности}
\newcommand{\department}{Высшая школа программной инженерии}
\newcommand{\city}{Санкт-Петербург}
\newcommand{\yearOf}{2025}

\newcommand{\doctype}{Курсовая работа}
\newcommand{\titleRussian}{Методы имитационного моделирования}
\newcommand{\subject}{Архитектура программных систем} 
\newcommand{\group}{5130904/30102}
\newcommand{\student}{Мальцев А.\,Л.}
\newcommand{\supervisor}{Дробинцев Д.\,Ф.}


% ---- Титульный лист ----
\begin{document}
\begin{titlepage}
    \begin{center}
        \large \university\\
        \faculty\\
        \department\\[5cm]


        \textbf{\LARGE \doctype} \\[0.5cm]
        \textbf{\Large \titleRussian} \\[0.5cm]
        по дисциплине <<\subject>> \\[0.5cm]
        вариант 12\\[4cm]
    \end{center}

    \vfill
    \begin{flushleft}
        \begin{tabular}{@{} p{0.65\textwidth} l @{}}
            Выполнил студент гр. \group & \student \\[0.25cm]
            Руководитель & \supervisor \\
        \end{tabular}
    \end{flushleft}

    
    \vfill
    \begin{center}
        \city, \yearOf\, г.
    \end{center}
\end{titlepage}

% ---- Нумерация со 2-й страницы ----
\setcounter{page}{2}

% ---- Оглавление ----
\tableofcontents
\newpage

% ---- Введение ----
\section{Введение}
\label{sec:intro}
В данной курсовой работе разработана имитационная модель системы массового обслуживания (СМО) для исследования характеристик вычислительной системы с множественными источниками заявок, буферной памятью и несколькими приборами обслуживания.

\textbf{Цель работы:} разработка и исследование программной модели СМО с заданными дисциплинами работы для анализа эффективности функционирования вычислительной системы.

\textbf{Задачи работы:}
\begin{itemize}
    \item Реализация имитационной модели СМО с использованием метода особых событий
    \item Исследование влияния входных параметров на выходные характеристики системы
    \item Синтез реальной вычислительной системы, соответствующей моделируемой СМО
    \item Определение оптимальной конфигурации системы, удовлетворяющей заданным требованиям
\end{itemize}

\textbf{Актуальность:} моделирование систем массового обслуживания позволяет оптимизировать конфигурацию вычислительных систем, минимизировать отказы в обслуживании и обеспечить эффективное использование ресурсов без необходимости дорогостоящих экспериментов на реальном оборудовании.

% ---- Постановка задачи ----
\section{Постановка задачи}
\label{sec:problem}

\textbf{Формализованная формула варианта:}
\[
\text{ИБ-ИЗ2-ПЗ1-Д10З3-Д10О4-Д2П2-Д2Б3-ОР1-ОД3}
\]

\subsection{Расшифровка формулы}
\noindent \textbf{Источники}
\begin{itemize}
    \item ИБ — бесконечный источник.
    \item ИЗ2 — равномерный закон генерации заявок (постоянный интервал между заявками).
\end{itemize}

\noindent \textbf{Приборы}
\begin{itemize}
    \item ПЗ1 — экспоненциальный закон распределения времени обслуживания.
\end{itemize}

\noindent \textbf{Описание дисциплин постановки и выбора}
\begin{itemize}
    \item Д10З3 — дисциплина буферизации: первое свободное место в буфере.
    \item Д10О4 — дисциплина отказа: выбивание последней поступившей заявки при переполнении буфера.
\end{itemize}

\noindent \textbf{Дисциплины постановки на обслуживание}
\begin{itemize}
    \item Д2П2 — дисциплина выбора прибора: выбор прибора по кольцу.
    \item Д2Б3 — дисциплина выбора из буфера: выбор заявки из буфера по кольцу.
\end{itemize}

\noindent \textbf{Виды отображения результатов работы программной модели}
\begin{itemize}
    \item ОР1 - отображение результатов: сводная таблица результатов.
    \item ОД3 - отображение динамики функционирования модели: временные диаграммы, текущее состояние.
\end{itemize}


% ---- Формализованная схема ----
\section{Формализованная схема и описание СМО}
\label{sec:scheme}
\begin{figure}[H]
    \centering
    \includegraphics[width=0.8\textwidth]{scheme.png}
    \caption{Формализованная схема системы массового обслуживания.}
    \label{fig:scheme}
\end{figure}

\textbf{Описание элементов системы:}

\begin{itemize}
    \item И\textsubscript{i} ($i = 1..n$) — \textit{источники заявок}. Каждый источник генерирует поток заявок, характеризующихся временем поступления и номером источника. Совокупность всех источников формирует суммарный входной поток системы.
    
    \item ДП — \textit{диспетчер постановки заявок}. Он определяет дальнейшую судьбу каждой вновь поступившей заявки: при наличии свободных приборов направляет её непосредственно на обслуживание, при их отсутствии — помещает в буферную память. Если буфер заполнен, диспетчер реализует дисциплину отказа или выбивания заявки из буфера.
    
    \item БП — \textit{буферная память}, предназначенная для временного хранения заявок, ожидающих обслуживания. Заполнение буфера происходит в соответствии с заданной дисциплиной постановки заявок.
    
    \item ДВ — \textit{диспетчер выбора заявок}. Он осуществляет выборку заявок из буфера и их распределение по свободным приборам согласно выбранной дисциплине обслуживания.
    
    \item П\textsubscript{j} ($j = 1..m$) — \textit{приборы обслуживания}. Каждый прибор выполняет обработку заявок, формируя выходной поток обслуженных требований. После завершения обслуживания прибор освобождается и готов к приёму следующей заявки.
\end{itemize}
\newpage

% ---- Временная диаграмма ----
\section{Временная диаграмма функционирования}
\label{sec:waveform}
Временная диаграмма строится методом особых событий, где каждое событие (поступление заявки или завершение обслуживания) отмечается на временной оси.

\begin{figure}[H]
    \centering
    \includegraphics[width=\textwidth]{waveform.png}
    \caption{Временная диаграмма функционирования СМО (метод особых событий).}
    \label{fig:wave}
\end{figure}

\textbf{Характерные моменты на диаграмме:}

\begin{enumerate}
    \item \textbf{Моменты поступления заявок:} отмечаются вертикальными линиями на временной оси для каждого источника. Заявки поступают с постоянными интервалами $\tau_i$.
    
    \item \textbf{Постановка в буфер:} если все приборы заняты, заявка размещается в первое свободное место буфера. На диаграмме это отображается как переход заявки в состояние ожидания в буфере.
    
    \item \textbf{Отказ (выбивание):} при переполнении буфера последняя поступившая заявка вытесняется. На диаграмме это отмечается как разрыв линии заявки и переход в состояние отказа.
    
    \item \textbf{Выбор на обслуживание:} следующая заявка из буфера выбирается по кольцу. На диаграмме это соответствует переходу из состояния ожидания в состояние обслуживания.
    
    \item \textbf{Освобождение прибора:} завершение обслуживания заявки отмечается как момент выхода заявки из системы. Прибор освобождается и может принять следующую заявку из буфера.
\end{enumerate}

Диаграмма демонстрирует работу всех дисциплин: заполнение буфера по порядку (Д10З3), выбивание последней заявки при переполнении (Д10О4), кольцевой выбор приборов (Д2П2) и заявок из буфера (Д2Б3).
\newpage

% ---- Второй этап ----
\section{Разработка и отладка программы}
\label{sec:stage2}

\subsection{Обобщённая блок-схема}
\begin{figure}[H]
    \centering
    \includegraphics[width=\textwidth]{flowchart.png}
    \caption{Блок-схема, описывающая схему функционирования СМО.}
    \label{fig:wave}
\end{figure}

\subsection{Описание модели и подход к реализации}
Моделирование реализовано методом особых событий с использованием календаря событий (event calendar) — приоритетной очереди событий, упорядоченной по времени.

\textbf{Календарь событий:} реализован как приоритетная очередь (min-heap), хранящая события типа (время, тип события, заявка, прибор, источник). События извлекаются в порядке возрастания времени.

\textbf{Обработка события поступления:}
\begin{enumerate}
    \item Создание новой заявки с уникальным ID и временем поступления.
    \item Поиск свободного прибора по кольцу.
    \item Если прибор найден: начало обслуживания, генерация времени обслуживания, планирование события завершения.
    \item Если приборов нет: попытка размещения в буфер (первое свободное место), при переполнении — вытеснение последней заявки.
    \item Планирование следующего поступления от того же источника.
\end{enumerate}

\textbf{Обработка события завершения обслуживания:}
\begin{enumerate}
    \item Завершение обслуживания на приборе.
    \item Запись статистики: время пребывания, время ожидания, время обслуживания.
    \item Если буфер не пуст: выбор заявки из буфера по кольцу, начало обслуживания.
\end{enumerate}

\subsection{Законы распределения}
\textbf{Равномерный закон генерации (ИЗ2):} заявки генерируются с постоянным интервалом $\tau_i$. Время следующего поступления:
\[
t_{next} = t_{current} + \tau
\]
где $\tau$ — интервал генерации для данного источника, задаётся константой.

\textbf{Экспоненциальный закон обслуживания (ПЗ1):} время обслуживания $T_{обсл}$ распределено по экспоненциальному закону с параметром $\mu$:
\[
f(t) = \mu e^{-\mu t}, \quad t \geq 0
\]
\[
F(t) = 1 - e^{-\mu t}
\]

Генерация случайной величины по методу обратной функции:
\[
T_{обсл} = -\frac{1}{\mu} \ln(1 - U) = -\frac{1}{\mu} \ln(U')
\]
где $U$ — случайная величина, равномерно распределённая на $[0, 1]$, $U' = 1 - U$ также равномерно распределена на $[0, 1]$.

В программе используется стандартная библиотека C++: \texttt{std::exponential\_distribution<double>} с параметром $\mu$.

\subsection{Динамическое и автоматическое отображение}
\textbf{ОД3 (динамическое отображение — временные диаграммы):} реализовано в виде виджета \texttt{TimelineWidget}, который отображает временную диаграмму функционирования системы в реальном времени. Диаграмма показывает:
\begin{itemize}
    \item Моменты поступления заявок от каждого источника
    \item Время пребывания заявок в буфере
    \item Время обслуживания заявок на приборах
    \item Моменты отказов (выбивания заявок)
    \item Состояние системы в каждый момент времени
\end{itemize}

Диаграмма обновляется при каждом шаге моделирования и позволяет визуально отслеживать работу системы.

\textbf{ОР1 (автоматическое отображение — сводная таблица результатов):} реализовано в виде виджета \texttt{AnalyticsWidget}, который отображает таблицы результатов после завершения моделирования:

\begin{itemize}
    \item \textbf{Таблица 1:} характеристики источников (количество заявок, вероятность отказа, среднее время пребывания, среднее время ожидания, среднее время обслуживания, дисперсии)
    \begin{center}
	\begin{tabular}{|c|c|c|c|c|c|c|c|}
		\hline
		\multicolumn{8}{|c|}{Характеристики источников ВС}                                                                                                   \\
		\hline
		№ источника & Количество заявок & $p_\text{отк}$ & $T_\text{преб}$ & $T_\text{БП}$ & $T_\text{обсл}$ & $\text{Д}_\text{БП}$ & $\text{Д}_\text{обсл}$ \\
		\hline
		И1          &                   &                &                 &               &                 &                      &                        \\
		\hline
		И1          &                   &                &                 &               &                 &                      &                        \\
		\hline
		...         &                   &                &                 &               &                 &                      &                        \\
		\hline
		Иn          &                   &                &                 &               &                 &                      &                        \\
		\hline
	\end{tabular}
    \end{center}
    
    \item \textbf{Таблица 2:} характеристики приборов (коэффициент использования)
    \begin{center}
	\begin{tabular}{|c|c|}
		\hline
		\multicolumn{2}{|c|}{Характеристики приборов ВС} \\
		\hline
		№ прибора & Коэффициент использования            \\
		\hline
		П1        &                                      \\
		\hline
		П2        &                                      \\
		\hline
		...       &                                      \\
		\hline
		Пm        &                                      \\
		\hline
	\end{tabular}
\end{center}
\end{itemize}

Также реализован виджет \texttt{EventCalendarWidget} для отображения календаря событий в пошаговом режиме, показывающий запланированные события и текущее состояние системы.
\begin{center}
  \begin{tabular}{|c|c|c|c|}
      \hline
      \multicolumn{2}{|c|}{Календарь событий} & Число заявок & Число отказов   \\
      \hline
      Событие                                 & Время        &               & \\
      \hline
      И1                                      &              &               & \\
      \hline
      ...                                     &              &               & \\
      \hline
      Иn                                      &              &               & \\
      \hline
      П1                                      &              &               & \\
      \hline
      ...                                     &              &               & \\
      \hline
      Пm                                      &              &               & \\
      \hline
  \end{tabular}
\end{center}

% ---------------------- СТРУКТУРА ПРОЕКТА ----------------------
\section{Структура проекта}
\label{sec:project_structure}

\textbf{Основные компоненты:}
\begin{itemize}
    \item \textbf{Core (ядро моделирования):} основные классы и алгоритмы моделирования (\texttt{Simulator}, \texttt{EventCalendar}, \texttt{Buffer}, \texttt{Device}, \texttt{Request}, \texttt{Event}, \texttt{Metrics}). Эти классы находятся в библиотеке \texttt{sim\_core} (папка \texttt{libs/sim\_core/}).
    \item \textbf{GUI (графический интерфейс):} Qt-виджеты и окно приложения — \texttt{MainWindow}, \texttt{TimelineWidget}, \texttt{EventCalendarWidget}, \texttt{AnalyticsWidget}. Код расположен в \texttt{apps/gui/} и использует API ядра для визуализации динамики и вывода аналитики.
    \item \textbf{CLI (консольный интерфейс и утилиты):} консольные утилиты для отладки и пакетных расчётов: пошаговый отладочный запуск (\texttt{apps/cli/src/main.cpp}) и runner для перебора конфигураций (\texttt{apps/cli/src/sim\_sweep.cpp}).
\end{itemize}

\subsection{Модульная структура и описание модулей}
\textbf{Ядро моделирования (\texttt{libs/sim\_core/}):}
\begin{itemize}
    \item \textbf{\texttt{Simulator.h/cpp}:} основной класс симулятора, управляющий циклом обработки событий методом особых событий. Отвечает за создание источников, приборов, запуск обслуживания на приборах, работу диспетчеров и взаимодействие с календарём событий.
    \item \textbf{\texttt{Event.h/cpp}:} представление события (время, тип события — поступление/завершение, связанная заявка/прибор/источник).
    \item \textbf{\texttt{EventCalendar.h/cpp}:} календарь событий — min-heap / приоритетная очередь для планирования и извлечения следующих по времени событий.
    \item \textbf{\texttt{Buffer.h/cpp}:} реализация буфера фиксированной ёмкости с дисциплинами постановки и выбора: Д10З3 (первое свободное место) и Д2Б3 (кольцевой выбор). Методы: \texttt{placeRequest()}, \texttt{takeRequest()}, \texttt{displaceRequest()}.
    \item \textbf{\texttt{Device.h/cpp} и \texttt{DevicePool.h/cpp}:} представление прибора обслуживания, состояние (свободен/занят), учёт времени занятости, методы \texttt{startService()}, \texttt{finishService()}, \texttt{isFree()} и пул приборов с поддержкой дисциплины выбора Д2П2 (по кольцу).
    \item \textbf{\texttt{Request.h/cpp}:} класс заявки с идентификатором, номером источника, временем поступления, временем начала и окончания обслуживания.
    \item \textbf{\texttt{Metrics.h/cpp}:} сбор статистики по системе и по каждому источнику/прибору: число поступивших, отказанных, завершённых, суммарные и средние времена, дисперсии, коэффициенты использования приборов.
    \item \textbf{Дополнительные модули:} \texttt{SourcePool} (управление источниками), \texttt{EventDispatcher} (правила постановки/выбора), и интерфейсы наблюдателей (Observers) для GUI/логирования.
\end{itemize}

\textbf{Графический интерфейс (\texttt{apps/gui/}):}
\begin{itemize}
    \item \textbf{\texttt{MainWindow.h/cpp}:} окно приложения, элементы управления симуляцией (Step, Run, Pause, Reset), запуск и остановка симулятора, передача событий и статистики виджетам.
    \item \textbf{\texttt{TimelineWidget.h/cpp}:} визуализация временной диаграммы (ОД3) — отображение поступлений, ожидания в буфере, обслуживания на приборах и отказов.
    \item \textbf{\texttt{EventCalendarWidget.h/cpp}:} табличное представление календаря событий и ближайших планируемых событий.
    \item \textbf{\texttt{AnalyticsWidget.h/cpp}:} отображение итоговой сводной таблицы (ОР1) — характеристики источников и приборов.
\end{itemize}

\textbf{Консольный интерфейс и утилиты (\texttt{apps/cli/}):}
\begin{itemize}
    \item \textbf{\texttt{main.cpp}:} пошаговый режим отладки с выводом состояния симулятора в консоль.
    \item \textbf{\texttt{sim\_sweep.cpp}:} утилита пакетного перебора конфигураций, реализующая сеточный поиск по входным параметрам, запуск симуляции для каждой конфигурации, отбор по ограничениям и экспорт результатов в CSV (\texttt{sweep\_results.csv}).
    \item \textbf{CMake:} каждый исполняемый файл собирается отдельной целью (\texttt{sim\_cli}, \texttt{sim\_sweep}), подключающей статическую библиотеку ядра \texttt{sim\_core}.
\end{itemize}

\subsection{Формат входных и выходных данных}
\textbf{Входные данные:} конфигурация моделирования задаётся структурой \texttt{SimulationConfig} \\
(файл \texttt{sim/simulator/SimulationConfig.h}) и включает параметры:
\begin{itemize}
    \item ёмкость буфера (\texttt{buffer\_capacity});
    \item число поступлений/максимальное число заявок (\texttt{max\_arrivals});
    \item зерно генератора (\texttt{seed});
    \item список источников (\texttt{sources}): для каждого — \texttt{id}, параметр интервала, тип распределения (постоянный/экспоненциальный);
    \item список приборов (\texttt{devices}): для каждого — \texttt{id}, параметр обслуживания (параметр \texttt{mu} для экспоненциального распределения), тип распределения.
\end{itemize}

\textbf{Выходные данные:} через класс \texttt{Metrics} доступны агрегированные и по‑источниковые показатели:
\begin{itemize}
    \item общие характеристики: число поступивших, отказанных и завершённых заявок; вероятность отказа; средние времена пребывания/ожидания/обслуживания;
    \item по источникам: количество поступлений, вероятность отказа, среднее и дисперсии времени ожидания и обслуживания;
    \item по приборам: коэффициент использования (загруженность) для каждого прибора;
    \item в пакетном режиме (утилита \texttt{sim\_sweep}): CSV-файл \texttt{sweep\_results.csv} с колонками конфигурации, метриками и отметкой соответствия ограничениям (passes).
\end{itemize}

% -------------------------------------------------------
% ---------------------- ТРЕТИЙ ЭТАП ---------------------
% -------------------------------------------------------

\section{Пример реальной вычислительной системы (синтез модели)}

\subsection{Описание}

В данном разделе приводится пример потенциально существующей вычислительной системы (ВС), полностью соответствующей заданным дисциплинам и законам.

Цель — задать количественные значения входных параметров и сформулировать требования к выходным характеристикам, чтобы в дальнейшем оценивать соответствие модели этим требованиям.

\subsubsection{Определение эффективности ВС}

Эффективность системы оценивается по трём выходным характеристикам:
\item $p_{\text{отк}}$ — вероятность отказа в обслуживании;
    \item $T_{\text{преб}}$ — среднее время пребывания заявки в системе;
    \item $K_{\text{исп}}$ — коэффициент использования приборов.

Задание желаемых значений этих величин формирует требования к работе ВС.

\subsubsection{Подход к проектированию модели}

Модель строится путём фиксации входных параметров в допустимых диапазонах, исходя из физического смысла компонентов системы. Это позволяет сформировать архитектуру, удовлетворяющую заданным требованиям.

\subsection{Пример ВС, соответствующей модели}

\subsubsection{Описание ВС}

В качестве реальной вычислительной системы рассматривается \textbf{система агрегации телеметрии от распределённых датчиков}, обеспечивающая непрерывный сбор, временное буферирование и обработку данных (температуры, давления, вибрации, расхода, уровня жидкости и др.) на промышленном объекте. Каждый датчик периодически передаёт пакет фиксированного размера на центральный модуль обработки. Центральный модуль распределяет поступающие пакеты по вычислительным узлам (приборы обслуживания) для выполнения фильтрации, сглаживания, детекции аномалий и агрегации.

\subsubsection{Соответствие модели с ВС}

\begin{center}
\begin{tabular}{|p{0.22\linewidth}|p{0.30\linewidth}|p{0.42\linewidth}|}
\hline
\textbf{Элемент системы} & \textbf{Дисциплина / закон} & \textbf{Соответствие в ВС} \\
\hline
Источники & ИБ–ИЗ2 & Датчики работают непрерывно и передают пакеты (2 Кбайт) с фиксированным интервалом — что соответствует бесконечному источнику с равномерной генерацией \\
\hline
Приборы & ПЗ1 & Время обработки пакета случайно из-за вариативности данных и нагрузки, что моделируется экспоненциальным распределением \\
\hline
Буфер & — & Массив фиксированных ячеек в модуле агрегации (ring-buffer) \\
\hline
Постановка в буфер & Д10З3 & Новый пакет размещается в первое свободное место — минимизирует задержку вставки и упрощает реализацию в промышленных контроллерах \\
\hline
Отказ в буфере & Д10О4 & При переполнении вытесняется последняя добавленная заявка: новые сэмплы шумнее, а старые необходимы для усреднения \\
\hline
Выбор из буфера & Д2Б3 & Чтение заявок — кольцевым обходом, что соответствует аппаратной реализации ring-buffer \\
\hline
Выбор прибора & Д2П2 & При освобождении прибора следующая заявка направляется на следующий по кольцу модуль — обеспечивает равномерную загрузку \\
\hline
\end{tabular}
\end{center}

\subsubsection{Ограничения и требуемые характеристики}

Требования к выходным характеристикам:
\item $p_{\text{отк}} \le 0.10$ — не более 10\% потерянных пакетов;
    \item $T_{\text{преб}} \le 200$ мс — задержка не должна нарушать требования реального времени;
    \item $K_{\text{исп}} \ge 0.9$ — приборы должны быть загружены не менее чем на 90\%.

Входные параметры и их диапазоны:

\begin{center}
\begin{tabular}{|p{0.5\linewidth}|p{0.5\linewidth}|}
\hline
\textbf{Параметр} & \textbf{Диапазон} \\
\hline
Количество датчиков & 4–20 \\
Интервал генерации телеметрии & 50–500 мс (шаг 50 мс) \\
Количество приборов & 1–6 \\
Тип прибора & Тип 1 ($120$ мс), Тип 2 ($90$ мс), Тип 3 ($60$ мс) \\
Размер буфера & 8–40 ячеек (шаг 8) \\
Размер заявки & 2 Кбайт \\
\hline
\end{tabular}
\end{center}

Стоимость компонентов:

\begin{center}
\begin{tabular}{|p{0.3\linewidth}|p{0.4\linewidth}|p{0.2\linewidth}|}
\hline
\textbf{Компонент} & \textbf{Характеристики} & \textbf{Цена, руб.} \\
\hline
Прибор Тип 1 & среднее время 120 мс & 6000 \\
Прибор Тип 2 & среднее время 90 мс & 9000 \\
Прибор Тип 3 & среднее время 60 мс & 15000 \\
Блок буфера & +8 ячеек & 800 \\
\hline
\end{tabular}
\end{center}

% -------------------------------------------------------
% ---------------------- ЧЕТВЁРТЫЙ ЭТАП ------------------
% -------------------------------------------------------
% Лучше варьировать одновременно меньшее число параметров, зафискировав остальные по легенде

\section{Анализ конфигураций}

\subsection{Описание}

Цель этапа — исследовать работу модели при различных конфигурациях входных параметров, отсеять недопустимые по требованиям и выбрать оптимальную по минимальной стоимости.

Необходимо:
\begin{itemize}
    \item определить достаточное число заявок для заданной точности;
    \item провести моделирование по сетке параметров;
    \item отфильтровать конфигурации по ограничениям;
    \item выбрать конфигурацию с минимальной стоимостью среди допустимых.
\end{itemize}

\subsection{Определение оптимального числа заявок}

Требуемая точность:
\item относительная точность: $\delta = 0.1$ (10\%);
    \item доверительная вероятность: $\alpha = 0.9$ ($t_\alpha = 1.643$).

Число заявок определяется по формуле:
\[
N = \frac{t_\alpha^2 (1 - p)}{p \delta^2},
\]
где $p$ — оценка вероятности отказа.

\subsubsection{Пристрелка}

Алгоритм определения необходимого числа заявок («пристрелка»):
\begin{enumerate}
    \item Назначить начальное количество заявок $N_0 = 200$.
    \item Провести моделирование: пропустить $N_0$ заявок через систему, получить оценку вероятности отказа $p_0$.
    \item Вычислить новое значение $N_1$.
    \item Выполнить моделирование с $N_1$ заявками и получить новую оценку $p_1$.
    \item Проверить условие стабилизации:
    \[
    |p_1 - p_0| < 0.1 \cdot p_0.
    \]
    \item Если условие выполнено, принять $N = N_1$ как достаточное.  
    В противном случае положить $p_0 := p_1$ и повторить шаги 3–6.
\end{enumerate}

На практике для всех конфигураций принято фиксированное значение:
\[
N = 10000,
\]
что обеспечивает точность даже при малых $p_{\text{отк}}$.

\subsection{Теория для тестирования}

Варьируемые параметры:
\begin{itemize}
    \item количество датчиков: 4---20;
    \item интервал генерации: 50---500 мс (с шагом 50 мс);
    \item количество приборов: 1–6;
    \item тип прибора: 1, 2 или 3;
    \item размер буфера: 8---40 (с шагом 8).
\end{itemize}

Общее число конфигураций — несколько тысяч. Для каждой:
\begin{enumerate}
    \item проводится моделирование с $N = 10000$ заявок;
    \item вычисляются $p_{\text{отк}}$, $T_{\text{преб}}$, $K_{\text{исп}}$;
    \item проверяется выполнение ограничений;
    \item вычисляется стоимость $C_{\text{сист}}$.
\end{enumerate}

Стоимость системы:
\[
C_{\text{сист}} = (\text{число приборов}) \cdot (\text{стоимость данного типа приборов}) + \frac{K}{8} \cdot 800.
\]

% ---- Анализ результатов ----
\section{Анализ результатов и выбор оптимальной конфигурации}
\label{sec:analysis}
\textbf{Процесс фильтрации конфигураций:}

Проведён полный перебор всех \textbf{15\,300} возможных конфигураций вычислительной системы. Отбор допустимых решений осуществлялся последовательно по трём критериям эффективности:

\begin{enumerate}
    \item Вероятность отказа не превышает 10\%: $p_{\text{отк}} \le 0.10$ — прошли \textbf{6\,370} конфигураций;
    \item Среднее время пребывания заявки в системе не превышает 200 мс: $T_{\text{преб}} \le 200\ \text{мс}$ — осталось \textbf{4\,938} конфигураций;
    \item Коэффициент использования приборов не ниже 90\%: $K_{\text{исп}} \ge 0.90$ — осталось \textbf{75} конфигураций.
\end{enumerate}

Из этих 75 удовлетворяющих всем условиям конфигураций отобраны 10 с наименьшей стоимостью. Их параметры и характеристики приведены в таблице~\ref{tab:top10}.

\begin{table}[h]
\centering
\caption{10 лучших конфигураций по критерию минимальной стоимости}
\label{tab:top10}
\small
\begin{tabular}{|c|c|c|c|c|c|c|c|c|}
\hline
\textbf{Датчики} & \textbf{Интервал, мс} & \textbf{Приборы} & \textbf{Тип прибора} & \textbf{Буфер} & $p_{\text{отк}}$ & $T_{\text{преб}}$, мс & $K_{\text{исп}}$ & \textbf{Стоимость, руб.} \\
\hline
5 & 150 & 3 & 2 & 8  & 0.0673 & 196.28 & 0.9337 & 27\,800 \\
7 & 200 & 3 & 2 & 8  & 0.0842 & 192.39 & 0.9262 & 27\,800 \\
4 & 100 & 5 & 1 & 8  & 0.0344 & 177.92 & 0.9044 & 30\,800 \\
5 & 150 & 2 & 3 & 8  & 0.0544 & 157.88 & 0.9131 & 30\,800 \\
6 & 150 & 5 & 1 & 8  & 0.0427 & 183.59 & 0.9040 & 30\,800 \\
8 & 200 & 5 & 1 & 8  & 0.0678 & 197.47 & 0.9043 & 30\,800 \\
5 & 100 & 6 & 1 & 8  & 0.0632 & 189.00 & 0.9478 & 36\,800 \\
7 & 150 & 4 & 2 & 8  & 0.0998 & 174.23 & 0.9430 & 36\,800 \\
7 & 150 & 6 & 1 & 16 & 0.0059 & 197.09 & 0.9076 & 37\,600 \\
8 & 200 & 4 & 2 & 16 & 0.0085 & 199.92 & 0.9039 & 37\,600 \\
\hline
\end{tabular}
\end{table}

Наилучшей по стоимости признана конфигурация:
\[
\text{датчики} = 5,\quad
\text{интервал} = 150\ \text{мс},\quad
\text{приборы} = 3,\quad
\text{тип прибора} = 2,\quad
\text{буфер} = 8\ \text{ячеек},
\]
с общей стоимостью \textbf{27\,800 руб.}.  
Для неё получены следующие характеристики:
\[
p_{\text{отк}} = 6.73\%,\qquad
T_{\text{преб}} = 196.3\ \text{мс},\qquad
K_{\text{исп}} = 93.4\%.
\]

\textbf{Анализ влияния параметров:}

\begin{itemize}
    \item \textbf{Увеличение числа приборов:} снижает вероятность отказа и время ожидания, увеличивает стоимость системы.
    \item \textbf{Увеличение ёмкости буфера:} снижает вероятность отказа при переполнении, незначительно влияет на стоимость.
    \item \textbf{Увеличение интенсивности обслуживания:} снижает время пребывания и вероятность отказа, значительно увеличивает стоимость (более производительные серверы).
    \item \textbf{Оптимальный баланс:} достигается при комбинации параметров, обеспечивающей выполнение ограничений при минимальной стоимости.
\end{itemize}

% ---- Выводы и рекомендации ----
\section{Вывод}
В ходе выполнения курсовой работы была разработана и исследована имитационная модель системы массового обслуживания с заданными дисциплинами работы.

\textbf{Достигнутые цели:}
\begin{itemize}
    \item С использованием метода особых событий реализована имитационная модель СМО с заданными дисциплинами работы. 
    \item Реализован графический интерфейс, позволяющий пользователю производить гибкую конфигурацию системы, наблюдать за динамикой работы системы (временные диаграммы, календарь событий) и результатами моделирования (сводные таблицы).
    \item Проведено исследование влияния входных параметров на выходные характеристики системы.
\end{itemize}

\textbf{Оптимальные конфигурации:} в результате исследования найдены конфигурации системы, удовлетворяющие требованиям к вероятности отказа ($p_{отк} \le 0.10$), времени пребывания ($T_{преб} \le 200$ мс) и коэффициенту использования ($K_{исп} \ge 0.90$) при минимальной стоимости.
\newpage





% ---- Приложения ----
\appendix
\section{Приложение A: Формулы и определения}
\subsection{Расчёт числа реализаций $N$}
\[
N = \frac{t_{\alpha}^2 (1-p)}{p \delta^2}
\]
где $t_{\alpha} = 1.643$ для $\alpha = 0.9$, $\delta = 0.1$ (10\% точность), $p$ — вероятность отказа.

\subsection{Расчёт вероятности отказа}
\[
p_{отк} = \frac{m}{n}
\]
где $m$ — количество отказов, $n$ — количество поступивших заявок.

\subsection{Расчёт среднего времени пребывания}
\[
T_{\text{преб}} = T_{\text{БП}} + T_{\text{обсл}}
\]
где $T_{\text{БП}}$ — среднее время пребывания в буфере, $T_{\text{обсл}}$ — среднее время обслуживания.

\subsection{Расчёт коэффициента использования}
\[
K_{\text{исп}} = \frac{T_{\text{занятости}}}{T_{\text{общее}}}
\]
где $T_{занятости}$ — суммарное время занятости прибора, $T_{общее}$ — общее время моделирования.

\subsection{Расчёт загрузки системы}
\[
\rho = \frac{\sum_{i=1}^{n} \lambda_i}{\sum_{j=1}^{m} \mu_j}
\]
где $\lambda_i$ — интенсивность $i$-го источника, $\mu$ — интенсивность обслуживания прибора, $m$ — количество приборов.

\subsection{Расчёт дисперсии}
\[
D = \frac{1}{n} \sum_{i=1}^{n} x_i^2 - \left( \frac{1}{n} \sum_{i=1}^{n} x_i \right)^2 = \overline{x^2} - \overline{x}^2
\]
где $x_i$ — значения случайной величины, $n$ — количество значений.

\newpage
\appendix
\section{Приложение B: Скриншоты}

\begin{figure}[H]
    \centering
    \includegraphics[width=0.95\linewidth]{main_start.png}
    \caption{Главный экран перед запуском симуляции: текущая конфигурация, результаты работы в реальном времени, календарь событий и графики.}
    \label{fig:main_start}
\end{figure}

\begin{figure}[H]
    \centering
    \includegraphics[width=0.95\linewidth]{main_end.png}
    \caption{Главный экран после завершения симуляции.}
    \label{fig:main_end}
\end{figure}

\begin{figure}[H]
    \centering
    \includegraphics[width=0.95\linewidth]{analytics.png}
    \caption{Экран аналитики: итоговые метрики}
    \label{fig:analysis}
\end{figure}

\end{document}

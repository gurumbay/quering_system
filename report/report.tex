\documentclass[a4paper]{article}
\usepackage[utf8]{inputenc}
\usepackage[russian]{babel}
\usepackage{booktabs}
\usepackage{geometry}
\usepackage{fancyhdr}
\usepackage{titlesec}
\usepackage{setspace}
\usepackage{amsmath, amssymb}
\usepackage{graphicx}
\usepackage{float}
\usepackage{listings}
\usepackage{tabularx}
\usepackage{caption}
\usepackage{color}
\usepackage{tcolorbox}
\usepackage{hyperref}
\hypersetup{hidelinks}

% ---- Поля страницы ----
\geometry{
    left=20mm,
    right=20mm,
    top=20mm,
    bottom=20mm
}

% ---- Межстрочный интервал ----
\onehalfspacing

% ---- Колонтитулы ----
\pagestyle{fancy}
\fancyhf{}
\fancyfoot[C]{\thepage}

% ---- Стиль заголовков ----
\titleformat{\section}{\bfseries\large}{\thesection}{1em}{}
\titleformat{\subsection}{\bfseries\normalsize}{\thesubsection}{1em}{}
\titleformat{\subsubsection}{\normalsize}{\thesubsubsection}{1em}{}

% ---- Настройки листингов ----
\lstset{
    basicstyle=\footnotesize\sffamily, 
    numbers=left,               
    numberstyle=\tiny,           
    stepnumber=1,                   
    numbersep=5pt,                
    backgroundcolor=\color{white},
    showspaces=false,            
    showstringspaces=false,
    showtabs=false,             
    frame=single,              
    tabsize=2,                 
    captionpos=t,              
    breaklines=true,           
    breakatwhitespace=false, 
    escapeinside={\%*}{*)},  
    inputencoding=utf8,
    extendedchars=true,
    literate={а}{{\char224}}1 {б}{{\char225}}1 {в}{{\char226}}1 {г}{{\char227}}1 {д}{{\char228}}1
             {е}{{\char229}}1 {ё}{{\char184}}1 {ж}{{\char230}}1 {з}{{\char231}}1 {и}{{\char232}}1
             {й}{{\char233}}1 {к}{{\char234}}1 {л}{{\char235}}1 {м}{{\char236}}1 {н}{{\char237}}1
             {о}{{\char238}}1 {п}{{\char239}}1 {р}{{\char240}}1 {с}{{\char241}}1 {т}{{\char242}}1
             {у}{{\char243}}1 {ф}{{\char244}}1 {х}{{\char245}}1 {ц}{{\char246}}1 {ч}{{\char247}}1
             {ш}{{\char248}}1 {щ}{{\char249}}1 {ъ}{{\char250}}1 {ы}{{\char251}}1 {ь}{{\char252}}1
             {э}{{\char253}}1 {ю}{{\char254}}1 {я}{{\char255}}1 {А}{{\char192}}1 {Б}{{\char193}}1
             {В}{{\char194}}1 {Г}{{\char195}}1 {Д}{{\char196}}1 {Е}{{\char197}}1 {Ё}{{\char168}}1
             {Ж}{{\char198}}1 {З}{{\char199}}1 {И}{{\char200}}1 {Й}{{\char201}}1 {К}{{\char202}}1
             {Л}{{\char203}}1 {М}{{\char204}}1 {Н}{{\char205}}1 {О}{{\char206}}1 {П}{{\char207}}1
             {Р}{{\char208}}1 {С}{{\char209}}1 {Т}{{\char210}}1 {У}{{\char211}}1 {Ф}{{\char212}}1
             {Х}{{\char213}}1 {Ц}{{\char214}}1 {Ч}{{\char215}}1 {Ш}{{\char216}}1 {Щ}{{\char217}}1
             {Ъ}{{\char218}}1 {Ы}{{\char219}}1 {Ь}{{\char220}}1 {Э}{{\char221}}1 {Ю}{{\char222}}1
             {Я}{{\char223}}1
}

% ---- Метаданные (замените значения) ----
\newcommand{\university}{Санкт-Петербургский политехнический университет Петра Великого}
\newcommand{\faculty}{Институт компьютерных наук и кибербезопасности}
\newcommand{\department}{Высшая школа программной инженерии}
\newcommand{\city}{Санкт-Петербург}
\newcommand{\yearOf}{2025}

\newcommand{\doctype}{Курсовая работа}
\newcommand{\titleRussian}{Методы имитационного моделирования}
\newcommand{\subject}{Архитектура программных систем} 
\newcommand{\group}{5130904/30102}
\newcommand{\student}{Мальцев А.\,Л.}
\newcommand{\supervisor}{Дробинцев Д.\,Ф.}


% ---- Титульный лист ----
\begin{document}
\begin{titlepage}
    \begin{center}
        \large \university\\
        \faculty\\
        \department\\[5cm]


        \textbf{\LARGE \doctype} \\[0.5cm]
        \textbf{\Large \titleRussian} \\[0.5cm]
        по дисциплине <<\subject>> \\[4cm]
    \end{center}

    \vfill
    \begin{flushleft}
        \begin{tabular}{@{} p{0.65\textwidth} l @{}}
            Выполнил студент гр. \group & \student \\[0.25cm]
            Руководитель & \supervisor \\
        \end{tabular}
    \end{flushleft}

    
    \vfill
    \begin{center}
        \city, \yearOf\, г.
    \end{center}
\end{titlepage}

% ---- Нумерация со 2-й страницы ----
\setcounter{page}{2}

% ---- Оглавление ----
\tableofcontents
\newpage

% ---- Введение ----
\section{Введение}
\label{sec:intro}
В данной курсовой работе разработана имитационная модель системы массового обслуживания (СМО) для исследования характеристик вычислительной системы с множественными источниками заявок, буферной памятью и несколькими приборами обслуживания.

\textbf{Цель работы:} разработка и исследование программной модели СМО с заданными дисциплинами работы для анализа эффективности функционирования вычислительной системы.

\textbf{Задачи работы:}
\begin{itemize}
    \item Реализация имитационной модели СМО с использованием метода особых событий
    \item Исследование влияния входных параметров на выходные характеристики системы
    \item Определение оптимальной конфигурации системы, удовлетворяющей заданным требованиям
    \item Синтез реальной вычислительной системы, соответствующей моделируемой СМО
\end{itemize}

\textbf{Требования к точности:} относительная точность результатов моделирования $\delta = 10\%$, доверительная вероятность $\alpha = 0.9$ ($t_{\alpha} = 1.643$).

\textbf{Актуальность:} моделирование систем массового обслуживания позволяет оптимизировать конфигурацию вычислительных систем, минимизировать отказы в обслуживании и обеспечить эффективное использование ресурсов без необходимости дорогостоящих экспериментов на реальном оборудовании.

% ---- Постановка задачи ----
\section{Постановка задачи}
\label{sec:problem}

\textbf{Формализованная формула варианта:}
\[
\text{ИБ-ИЗ2-ПЗ1-Д10З3-Д10О4-Д2П2-Д2Б3-ОР1-ОД3}
\]

\textbf{Расшифровка формулы:}
\begin{itemize}
    \item \textbf{ИБ} — бесконечный источник (неограниченное количество заявок)
    \item \textbf{ИЗ2} — равномерный закон генерации заявок (постоянный интервал между заявками)
    \item \textbf{ПЗ1} — экспоненциальный закон распределения времени обслуживания
    \item \textbf{Д10З3} — дисциплина буферизации: первое свободное место в буфере
    \item \textbf{Д10О4} — дисциплина отказа: выбивание последней поступившей заявки при переполнении буфера
    \item \textbf{Д2П2} — дисциплина выбора прибора: выбор прибора по кольцу (round-robin)
    \item \textbf{Д2Б3} — дисциплина выбора из буфера: выбор заявки из буфера по кольцу
    \item \textbf{ОР1} — сводная таблица результатов
    \item \textbf{ОД3} — временные диаграммы
\end{itemize}

\textbf{Исходные параметры модели:}
\begin{itemize}
    \item \textbf{Число источников:} $n = 3$ (по умолчанию)
    \item \textbf{Тип источников:} бесконечные источники с равномерным законом генерации
    \item \textbf{Интервалы генерации заявок:} $\tau_1 = 3.0$, $\tau_2 = 4.0$, $\tau_3 = 5.0$ единиц времени
    \item \textbf{Интенсивности источников:} $\lambda_1 = 1/3 = 0.333$, $\lambda_2 = 1/4 = 0.25$, $\lambda_3 = 1/5 = 0.2$ заявок/единицу времени
    \item \textbf{Число приборов:} $m = 3$ (по умолчанию)
    \item \textbf{Интенсивность обслуживания:} $\mu = 0.3$ заявок/единицу времени (для всех приборов)
    \item \textbf{Ёмкость буфера:} $K = 3$ (по умолчанию)
    \item \textbf{Максимальное число реализаций:} $N_{max} = 1000$ (по умолчанию)
    \item \textbf{Seed для генератора случайных чисел:} $52$
\end{itemize}

% ---- Формализованная схема ----
\section{Формализованная схема и описание СМО}
\label{sec:scheme}
\begin{figure}[H]
    \centering
    %\includegraphics[width=0.8\textwidth]{scheme.pdf}
    \caption{Формализованная схема системы массового обслуживания.}
    \label{fig:scheme}
    \textit{Примечание: вставьте векторный рисунок формализованной схемы}
\end{figure}

\textbf{Описание элементов системы:}

\textbf{Источники заявок (ИБ-ИЗ2):} Система содержит $n$ бесконечных источников, каждый из которых генерирует заявки с постоянным интервалом $\tau_i$ (равномерный закон). Интенсивность генерации $i$-го источника: $\lambda_i = 1/\tau_i$. Источники работают независимо и непрерывно до достижения максимального числа заявок $N_{max}$.

\textbf{Буферная память (Д10З3, Д10О4):} Буфер представляет собой очередь ограниченной ёмкости $K$. При поступлении заявки выбирается первое свободное место в буфере (Д10З3). При переполнении буфера последняя поступившая заявка вытесняется (Д10О4), освобождая место для новой.

\textbf{Диспетчер выбора прибора (Д2П2):} При освобождении прибора следующая заявка из буфера направляется на следующий по кольцу прибор (round-robin), начиная с последнего использованного прибора.

\textbf{Диспетчер выбора из буфера (Д2Б3):} Заявки из буфера выбираются последовательно по кольцу, начиная с указателя на первую занятую позицию.

\textbf{Приборы обслуживания (ПЗ1):} Система содержит $m$ идентичных приборов. Время обслуживания заявки распределено по экспоненциальному закону с параметром $\mu$: $f(t) = \mu e^{-\mu t}$, $t \geq 0$. Среднее время обслуживания: $T_{обсл} = 1/\mu$.

% ---- Временная диаграмма ----
\section{Временная диаграмма функционирования}
\label{sec:waveform}
Временная диаграмма строится методом особых событий, где каждое событие (поступление заявки, начало обслуживания, завершение обслуживания, отказ) отмечается на временной оси.

\begin{figure}[H]
    \centering
    %\includegraphics[width=\textwidth]{waveform.pdf}
    \caption{Временная диаграмма функционирования СМО (метод особых событий).}
    \label{fig:wave}
    \textit{Примечание: вставьте векторный рисунок временной диаграммы}
\end{figure}

\textbf{Характерные моменты на диаграмме:}

\begin{enumerate}
    \item \textbf{Моменты поступления заявок:} отмечаются вертикальными линиями на временной оси для каждого источника. Заявки поступают с постоянными интервалами $\tau_i$.
    
    \item \textbf{Постановка в буфер:} если все приборы заняты, заявка размещается в первое свободное место буфера. На диаграмме это отображается как переход заявки в состояние ожидания в буфере.
    
    \item \textbf{Отказ (выбивание):} при переполнении буфера последняя поступившая заявка вытесняется. На диаграмме это отмечается как разрыв линии заявки и переход в состояние отказа.
    
    \item \textbf{Выбор на обслуживание:} при освобождении прибора заявка из буфера выбирается по кольцу и направляется на следующий по кольцу прибор. На диаграмме это соответствует переходу из состояния ожидания в состояние обслуживания.
    
    \item \textbf{Освобождение прибора:} завершение обслуживания заявки отмечается как момент выхода заявки из системы. Прибор освобождается и может принять следующую заявку из буфера.
\end{enumerate}

Диаграмма демонстрирует работу всех дисциплин: заполнение буфера по порядку (Д10З3), выбивание последней заявки при переполнении (Д10О4), циклический выбор приборов (Д2П2) и заявок из буфера (Д2Б3).

% ---- Первый этап ----
\section{Первый этап}
\label{sec:stage1}
\subsection{Бланк задания с заполненными исходными данными}
\begin{table}[H]
\centering
\caption{Бланк задания}
\begin{tabular}{|l|l|}
\hline
\textbf{Параметр} & \textbf{Значение} \\
\hline
Вариант & ИБ-ИЗ2-ПЗ1-Д10З3-Д10О4-Д2П2-Д2Б3-ОР1-ОД3 \\
\hline
Тип источника & ИБ (бесконечный) \\
\hline
Закон генерации & ИЗ2 (равномерный) \\
\hline
Число источников & 3 \\
\hline
Интервалы генерации & $\tau_1 = 3.0$, $\tau_2 = 4.0$, $\tau_3 = 5.0$ \\
\hline
Интенсивности источников & $\lambda_1 = 0.333$, $\lambda_2 = 0.25$, $\lambda_3 = 0.2$ \\
\hline
Закон обслуживания & ПЗ1 (экспоненциальный) \\
\hline
Число приборов & 3 \\
\hline
Интенсивность обслуживания & $\mu = 0.3$ \\
\hline
Ёмкость буфера & $K = 3$ \\
\hline
Дисциплина буферизации & Д10З3 (первое свободное место) \\
\hline
Дисциплина отказа & Д10О4 (выбивание последней поступившей) \\
\hline
Дисциплина выбора прибора & Д2П2 (по кольцу) \\
\hline
Дисциплина выбора из буфера & Д2Б3 (по кольцу) \\
\hline
Отображение результатов & ОР1 (сводная таблица) \\
\hline
Отображение динамики & ОД3 (временные диаграммы) \\
\hline
\end{tabular}
\end{table}

\subsection{Краткие ответы на контрольные вопросы}
\begin{enumerate}
    \item \textbf{Типы источников и их особенности:}
    \begin{itemize}
        \item \textbf{ИБ (бесконечный источник):} генерирует неограниченное количество заявок до достижения максимального числа реализаций $N_{max}$. Источники работают независимо и непрерывно.
        \item \textbf{ИЗ2 (равномерный закон):} заявки генерируются с постоянным интервалом $\tau_i$. Интенсивность генерации $\lambda_i = 1/\tau_i$ постоянна для каждого источника.
    \end{itemize}
    
    \item \textbf{Принципы построения моделирующего алгоритма:}
    \begin{itemize}
        \item Использован метод \textbf{особых событий} (event-driven simulation), где моделирование ведётся по моментам наступления событий: поступление заявки, начало обслуживания, завершение обслуживания.
        \item События хранятся в календаре событий (приоритетной очереди), упорядоченной по времени.
        \item Преимущества метода особых событий: точность моделирования, отсутствие ошибок дискретизации, эффективность при редких событиях.
        \item Альтернативный метод Delta-T не использован, так как требует постоянного шага по времени и менее эффективен при редких событиях.
    \end{itemize}
    
    \item \textbf{Описание дисциплин:}
    \begin{itemize}
        \item \textbf{Д10З3 (дисциплина буферизации):} при постановке заявки в буфер выбирается первое свободное место, начиная с индекса 0. Заявка размещается в найденный свободный слот.
        \item \textbf{Д10О4 (дисциплина отказа):} при переполнении буфера последняя поступившая заявка вытесняется, освобождая место для новой заявки. Вытесненная заявка получает отказ в обслуживании.
        \item \textbf{Д2П2 (дисциплина выбора прибора):} при освобождении прибора заявка из буфера направляется на следующий по кольцу прибор (round-robin). Указатель на следующий прибор циклически смещается.
        \item \textbf{Д2Б3 (дисциплина выбора из буфера):} заявки из буфера выбираются последовательно по кольцу, начиная с указателя на первую занятую позицию. После выбора заявки указатель смещается на следующую позицию.
    \end{itemize}
\end{enumerate}

% ---- Второй этап ----
\section{Второй этап: Разработка и отладка программы}
\label{sec:stage2}
\subsection{Описание модели и подход к реализации}
Моделирование реализовано методом особых событий с использованием календаря событий (event calendar) — приоритетной очереди событий, упорядоченной по времени.

\textbf{Структура алгоритма:}
\begin{lstlisting}[language=C++,caption={Псевдокод основного цикла моделирования}]
while (календарь не пуст) {
    событие = извлечь_следующее_событие_из_календаря();
    текущее_время = время_события;
    
    switch (тип_события) {
        case ПОСТУПЛЕНИЕ_ЗАЯВКИ:
            создать_заявку();
            записать_статистику_поступления();
            if (есть_свободный_прибор) {
                начать_обслуживание();
            } else {
                попытаться_поместить_в_буфер();
            }
            запланировать_следующее_поступление();
            break;
            
        case ЗАВЕРШЕНИЕ_ОБСЛУЖИВАНИЯ:
            завершить_обслуживание();
            записать_статистику_завершения();
            if (буфер_не_пуст) {
                взять_заявку_из_буфера();
                начать_обслуживание();
            }
            break;
    }
    
    if (симуляция_завершена()) break;
}
\end{lstlisting}

\textbf{Календарь событий:} реализован как приоритетная очередь (min-heap), хранящая события типа (время, тип события, ID заявки, ID прибора, ID источника). События извлекаются в порядке возрастания времени.

\textbf{Обработка события поступления:}
\begin{enumerate}
    \item Создание новой заявки с уникальным ID и временем поступления
    \item Запись статистики поступления для источника
    \item Поиск свободного прибора по кольцу (round-robin)
    \item Если прибор найден: начало обслуживания, генерация времени обслуживания, планирование события завершения
    \item Если приборов нет: попытка размещения в буфер (первое свободное место), при переполнении — вытеснение последней заявки
    \item Планирование следующего поступления от того же источника
\end{enumerate}

\textbf{Обработка события завершения обслуживания:}
\begin{enumerate}
    \item Завершение обслуживания на приборе
    \item Запись статистики: время пребывания, время ожидания, время обслуживания
    \item Если буфер не пуст: выбор заявки из буфера по кольцу, начало обслуживания
\end{enumerate}

\subsection{Законы распределения}
\textbf{Равномерный закон генерации (ИЗ2):} заявки генерируются с постоянным интервалом $\tau_i$. Время следующего поступления:
\[
t_{next} = t_{current} + \tau_i
\]
где $\tau_i$ — интервал генерации для $i$-го источника, задаётся константой.

\textbf{Экспоненциальный закон обслуживания (ПЗ1):} время обслуживания $T_{обсл}$ распределено по экспоненциальному закону с параметром $\mu$:
\[
f(t) = \mu e^{-\mu t}, \quad t \geq 0
\]
\[
F(t) = 1 - e^{-\mu t}
\]

Генерация случайной величины по методу обратной функции:
\[
T_{обсл} = -\frac{1}{\mu} \ln(1 - U) = -\frac{1}{\mu} \ln(U')
\]
где $U$ — случайная величина, равномерно распределённая на $[0, 1]$, $U' = 1 - U$ также равномерно распределена на $[0, 1]$.

В программе используется стандартная библиотека C++: \texttt{std::exponential\_distribution<double>} с параметром $\mu$.

\subsection{Динамическое и автоматическое отображение}
\textbf{ОД3 (динамическое отображение — временные диаграммы):} реализовано в виде виджета \texttt{TimelineWidget}, который отображает временную диаграмму функционирования системы в реальном времени. Диаграмма показывает:
\begin{itemize}
    \item Моменты поступления заявок от каждого источника
    \item Время пребывания заявок в буфере
    \item Время обслуживания заявок на приборах
    \item Моменты отказов (выбивания заявок)
    \item Состояние системы в каждый момент времени
\end{itemize}

Диаграмма обновляется при каждом шаге моделирования и позволяет визуально отслеживать работу системы.

\textbf{ОР1 (автоматическое отображение — сводная таблица результатов):} реализовано в виде виджета \texttt{AnalyticsWidget}, который отображает таблицы результатов после завершения моделирования:

\begin{itemize}
    \item \textbf{Таблица 1:} характеристики источников (количество заявок, вероятность отказа, среднее время пребывания, среднее время ожидания, среднее время обслуживания, дисперсии)
    \item \textbf{Таблица 2:} характеристики приборов (коэффициент использования)
\end{itemize}

Также реализован виджет \texttt{EventCalendarWidget} для отображения календаря событий в пошаговом режиме, показывающий запланированные события и текущее состояние системы.

% ---- Третий этап ----
\section{Третий этап: Пример реальной ВС (синтез модели)}
\label{sec:stage3}
\subsection{Описание конкретной технической системы}
Рассматривается система обработки изображений в реальном времени для автоматизированной сортировки товаров на конвейере.

\textbf{Источники заявок (ИБ-ИЗ2):} видеокамеры, установленные над конвейером. Каждая камера снимает конвейер с постоянной частотой кадров (равномерный закон генерации). Камера генерирует заявки непрерывно до выключения системы (бесконечный источник). Интервалы генерации определяются частотой съёмки камеры: $\tau_1 = 3.0$ мс, $\tau_2 = 4.0$ мс, $\tau_3 = 5.0$ мс.

\textbf{Приборы обслуживания (ПЗ1):} GPU-серверы для обработки изображений с нейросетевыми моделями распознавания. Время обработки варьируется в зависимости от сложности сцены, количества объектов на изображении, состояния GPU, что приводит к экспоненциальному распределению времени обслуживания с параметром $\mu = 0.3$ обработанных кадров/мс.

\textbf{Буферная память:} очередь заданий в оперативной памяти, хранящая ссылки на изображения. Ёмкость буфера: $K = 3$ изображения.

\textbf{Дисциплины работы:}
\begin{itemize}
    \item \textbf{Д10З3:} изображения размещаются в первое свободное место буфера — простая и эффективная реализация для FIFO-обработки кадров.
    \item \textbf{Д10О4:} при переполнении буфера последнее поступившее изображение вытесняется — последний кадр часто дублирует предыдущий, потеря менее критична для непрерывности обработки.
    \item \textbf{Д2П2:} при освобождении сервера следующее изображение направляется на следующий по кольцу GPU-сервер — балансировка нагрузки между серверами.
    \item \textbf{Д2Б3:} изображения из буфера выбираются последовательно по кругу — равномерная обработка кадров со всех камер без приоритетов.
\end{itemize}

\textbf{Соответствие дисциплин и законов распределения:}
\begin{itemize}
    \item Равномерный закон генерации соответствует постоянной частоте съёмки камер
    \item Экспоненциальный закон обслуживания отражает вариативность сложности обработки изображений
    \item Дисциплины буферизации и выбора обеспечивают справедливое распределение нагрузки
\end{itemize}

\subsection{Требования к результатам и ограничения}
\textbf{Целевые значения выходных характеристик:}
\begin{itemize}
    \item \textbf{Вероятность отказа:} $p_{отк} \le 0.05$ (5\%) — потеря кадров должна быть минимальной для непрерывности контроля конвейера
    \item \textbf{Среднее время пребывания:} $T_{преб} \le 200$ мс — требование реального времени, критично для оперативного реагирования
    \item \textbf{Коэффициент использования приборов:} $K_{исп} \ge 0.70$ (70\%) — эффективное использование дорогостоящего GPU-оборудования
\end{itemize}

\textbf{Входные параметры для исследования:}
\begin{itemize}
    \item Количество камер (источников): от 3 до 8
    \item Интервалы генерации: от 100 до 500 мс
    \item Количество GPU-серверов: от 2 до 5
    \item Интенсивность обслуживания: от 0.1 до 0.5 (1/мс)
    \item Ёмкость буфера: от 3 до 16 слотов
\end{itemize}

\textbf{Экономические ограничения:}
\begin{itemize}
    \item Стоимость GPU-сервера базового: 80,000 руб. (время обработки 100 мс)
    \item Стоимость GPU-сервера среднего: 120,000 руб. (время обработки 70 мс)
    \item Стоимость GPU-сервера производительного: 180,000 руб. (время обработки 50 мс)
    \item Расширение буфера: +5,000 руб. за каждые 4 дополнительных слота
\end{itemize}

Задача синтеза: подобрать конфигурацию системы (количество серверов, их производительность, ёмкость буфера), обеспечивающую выполнение требований к выходным характеристикам при минимальной стоимости.

% ---- Четвёртый этап ----
\section{Четвёртый этап: Исследование модели и оптимизация}
\label{sec:stage4}
\subsection{План экспериментов и сетка параметров}
\textbf{Пространство поиска:}

\begin{table}[H]
\centering
\caption{Диапазоны входных параметров}
\begin{tabular}{|l|c|c|c|}
\hline
\textbf{Параметр} & \textbf{Минимум} & \textbf{Максимум} & \textbf{Шаг} \\
\hline
Число источников & 3 & 6 & 1 \\
\hline
Интервал генерации (мс) & 2.0 & 6.0 & 0.5 \\
\hline
Число приборов & 2 & 5 & 1 \\
\hline
Интенсивность обслуживания $\mu$ & 0.1 & 0.5 & 0.05 \\
\hline
Ёмкость буфера & 3 & 16 & 2 \\
\hline
\end{tabular}
\end{table}

\textbf{План экспериментов:} для каждой комбинации параметров из сетки выполняется моделирование с определением необходимого числа реализаций $N$ по алгоритму «пристрелки», затем сбор статистики и расчёт выходных характеристик.

\textbf{Примеры конфигураций для исследования:}
\begin{itemize}
    \item Базовая: $n=3$, $\tau=[3.0, 4.0, 5.0]$, $m=3$, $\mu=0.3$, $K=3$
    \item Высокая нагрузка: $n=5$, $\tau=[2.0, 2.5, 3.0, 3.5, 4.0]$, $m=3$, $\mu=0.3$, $K=3$
    \item Увеличенный буфер: $n=3$, $\tau=[3.0, 4.0, 5.0]$, $m=3$, $\mu=0.3$, $K=8$
    \item Больше приборов: $n=3$, $\tau=[3.0, 4.0, 5.0]$, $m=5$, $\mu=0.3$, $K=3$
    \item Высокая производительность: $n=3$, $\tau=[3.0, 4.0, 5.0]$, $m=3$, $\mu=0.5$, $K=3$
\end{itemize}

\subsection{Определение числа реализаций (N)}
\textbf{Формула для оценки числа реализаций:}
\[
N = \frac{t_{\alpha}^2 (1-p)}{p \delta^2},
\]
где:
\begin{itemize}
    \item $t_{\alpha} = 1.643$ — квантиль стандартного нормального распределения для доверительной вероятности $\alpha = 0.9$
    \item $\delta = 0.1$ — относительная точность (10\%)
    \item $p$ — вероятность отказа в обслуживании
\end{itemize}

\textbf{Алгоритм «пристрелки» (итерационный процесс):}
\begin{enumerate}
    \item \textbf{Итерация 0:} Запуск моделирования с начальным числом заявок $N_0 = 100$
    \item Выполнение моделирования для $N_0$ заявок
    \item Расчёт выходных характеристик, включая вероятность отказа $p_0$
    \item Подстановка $p_0$ в формулу для получения $N_1 = \frac{1.643^2 (1-p_0)}{p_0 \cdot 0.1^2}$
    \item \textbf{Итерация 1:} Запуск моделирования с $N_1$ заявок
    \item Расчёт новых характеристик, включая $p_1$
    \item Проверка условия остановки: $|p_1 - p_0| < 0.1 \cdot p_0$ (разница менее 10\% от $p_0$)
    \item Если условие выполнено: $N = N_1$, завершение процесса
    \item Если условие не выполнено: переход к следующей итерации с $N_2 = \frac{1.643^2 (1-p_1)}{p_1 \cdot 0.1^2}$ и повторение процесса
\end{enumerate}

\textbf{Критерий завершения:} процесс завершается, когда разница между вероятностями отказа на двух последовательных итерациях становится менее 10\% от предыдущего значения:
\[
|p_i - p_{i-1}| < 0.1 \cdot p_{i-1}
\]

\textbf{Примечание:} данный алгоритм требует реализации автоматического процесса определения $N$. В текущей версии программы пользователь задаёт максимальное число заявок $N_{max}$ вручную, что позволяет выполнить моделирование, но не гарантирует достижение заданной точности автоматически.

\subsection{Критерий эффективности}
\textbf{Целевая функция:} минимизация стоимости системы при выполнении ограничений на выходные характеристики.

\textbf{Формулы расчёта:}

\textbf{Стоимость системы:}
\[
C_{системы} = m \cdot C_{сервера}(\mu) + \left\lfloor \frac{K - K_{базовый}}{4} \right\rfloor \cdot C_{буфера}
\]
где:
\begin{itemize}
    \item $m$ — количество серверов
    \item $C_{сервера}(\mu)$ — стоимость сервера в зависимости от интенсивности обслуживания $\mu$
    \item $K$ — ёмкость буфера
    \item $K_{базовый} = 3$ — базовая ёмкость буфера
    \item $C_{буфера} = 5000$ руб. — стоимость расширения буфера на 4 слота
\end{itemize}

\textbf{Стоимость серверов:}
\begin{itemize}
    \item $C_{сервера}(0.1) = 80000$ руб. (базовый, $\mu = 0.1$)
    \item $C_{сервера}(0.14) = 120000$ руб. (средний, $\mu = 0.14$)
    \item $C_{сервера}(0.2) = 180000$ руб. (производительный, $\mu = 0.2$)
\end{itemize}

\textbf{Ограничения:}
\begin{align}
p_{отк} &\le 0.05 \\
T_{преб} &\le 200 \text{ мс} \\
K_{исп} &\ge 0.70
\end{align}

\textbf{Ранжирование конфигураций:} конфигурации, удовлетворяющие ограничениям, ранжируются по возрастанию стоимости $C_{системы}$. Оптимальная конфигурация — минимальная стоимость при выполнении всех ограничений.

% ---- Описание программы (ПО) ----
\section{Описание программной реализации}
\label{sec:software}
\subsection{Обобщенная блок-схема}
\begin{figure}[H]
    \centering
    %\includegraphics[width=0.9\textwidth]{architecture.pdf}
    \caption{Блок-схема архитектуры программной системы}
    \label{fig:architecture}
    \textit{Примечание: вставьте блок-схему архитектуры}
\end{figure}

\textbf{Основные компоненты:}
\begin{itemize}
    \item \textbf{Core (ядро моделирования):} Simulator, EventCalendar, Buffer, Device, Request, Event, Metrics
    \item \textbf{GUI (графический интерфейс):} MainWindow, TimelineWidget, EventCalendarWidget, AnalyticsWidget
    \item \textbf{CLI (консольный интерфейс):} main.cpp для пошагового режима отладки
\end{itemize}

\subsection{Модульная структура и описание модулей}
\textbf{Ядро моделирования (include/sim/core/):}

\begin{itemize}
    \item \textbf{Simulator.h/cpp:} основной класс симулятора, реализует алгоритм моделирования методом особых событий. Управляет источниками, приборами, буфером и календарём событий.
    
    \item \textbf{Event.h/cpp:} класс события в системе. Содержит время события, тип (поступление/завершение обслуживания), ID заявки, прибора и источника.
    
    \item \textbf{EventCalendar.h/cpp:} календарь событий — приоритетная очередь для хранения и извлечения событий в порядке возрастания времени. Реализован как min-heap.
    
    \item \textbf{Buffer.h/cpp:} буферная память ограниченной ёмкости. Реализует дисциплины Д10З3 (первое свободное место) и Д2Б3 (выбор по кольцу). Методы: place\_request(), take\_request(), displace\_request().
    
    \item \textbf{Device.h/cpp:} класс прибора обслуживания. Отслеживает состояние (свободен/занят), текущую обслуживаемую заявку, время занятости. Методы: start\_service(), finish\_service(), is\_free().
    
    \item \textbf{Request.h/cpp:} класс заявки. Содержит ID заявки, ID источника, время поступления, время начала обслуживания.
    
    \item \textbf{Metrics.h/cpp:} класс сбора статистики. Собирает данные о поступлениях, отказах, завершениях обслуживания, времени пребывания, времени ожидания, времени обслуживания для каждого источника и прибора. Вычисляет вероятности, средние значения, дисперсии.
\end{itemize}

\textbf{Графический интерфейс (src/gui/):}

\begin{itemize}
    \item \textbf{MainWindow.h/cpp:} главное окно приложения. Управляет конфигурацией моделирования, элементами управления (Step, Run, Pause, Run to End, Reset), отображением результатов, интеграцией виджетов.
    
    \item \textbf{TimelineWidget.h/cpp:} виджет временной диаграммы (ОД3). Отображает временную диаграмму функционирования системы: поступления заявок, состояния буфера, обслуживание на приборах. Поддерживает масштабирование и прокрутку.
    
    \item \textbf{EventCalendarWidget.h/cpp:} виджет календаря событий. Отображает запланированные события и текущее состояние системы в табличном виде.
    
    \item \textbf{AnalyticsWidget.h/cpp:} виджет аналитики (ОР1). Отображает таблицы результатов: характеристики источников и приборов после завершения моделирования.
\end{itemize}

\textbf{Консольный интерфейс:}

\begin{itemize}
    \item \textbf{main.cpp:} консольное приложение для пошагового режима отладки. Позволяет выполнять моделирование пошагово с выводом состояния системы на каждом шаге.
\end{itemize}

\subsection{Формат входных/выходных данных}
\textbf{Входные данные:} конфигурация моделирования задаётся через структуру \texttt{SimulationConfig}:

\begin{lstlisting}[language=C++,caption={Структура конфигурации}]
struct SourceConfig {
    size_t id;
    double arrival_interval;  // интервал между заявками
};

struct SimulationConfig {
    size_t num_devices = 3;           // число приборов
    size_t buffer_capacity = 3;       // ёмкость буфера
    double device_intensity = 0.3;     // интенсивность обслуживания μ
    size_t max_arrivals = 1000;       // максимальное число заявок
    double max_time = 1e9;            // максимальное время моделирования
    uint32_t seed = 52;               // seed для ГСЧ
    std::vector<SourceConfig> sources; // источники
};
\end{lstlisting}

\textbf{Выходные данные:} результаты моделирования доступны через класс \texttt{Metrics}:

\begin{itemize}
    \item Общие характеристики: количество поступивших, отказанных, завершённых заявок; вероятность отказа; средние времена пребывания, ожидания, обслуживания
    \item Характеристики по источникам: количество заявок, вероятность отказа, среднее время пребывания, среднее время ожидания, среднее время обслуживания, дисперсии времени ожидания и обслуживания
    \item Характеристики по приборам: коэффициент использования каждого прибора
\end{itemize}

В GUI результаты отображаются в таблицах виджета \texttt{AnalyticsWidget}.

\subsection{Код: пример ключевой функции}
\begin{lstlisting}[language=C++,caption={Обработка события завершения обслуживания}]
void Simulator::handle_service_end(size_t device_id) {
    // Завершение обслуживания текущей заявки
    size_t finished_id = devices_[device_id].finish_service(current_time_);
    
    if (finished_id < requests_.size()) {
        const Request& req = requests_[finished_id];
        double time_in_system = current_time_ - req.get_arrival_time();
        double waiting_time = req.get_service_start_time() - req.get_arrival_time();
        double service_time = current_time_ - req.get_service_start_time();
        
        // Запись статистики завершения
        metrics_.record_completion(finished_id, req.get_source_id(), 
                                   time_in_system, waiting_time, service_time);
        metrics_.record_device_busy_time(device_id, service_time);
        metrics_.record_service_end_event(current_time_, finished_id,
                                          req.get_source_id(), device_id);
        
        end_time_ = current_time_;
    }
    
    // Д2Б3: выбор заявки из буфера по кольцу
    if (!buffer_.is_empty()) {
        auto [next_request, buffer_slot_index] = buffer_.take_request();
        if (next_request.has_value()) {
            size_t request_id = *next_request;
            metrics_.record_buffer_take_event(current_time_, request_id,
                requests_[request_id].get_source_id(), device_id, buffer_slot_index);
            
            // Начало обслуживания выбранной заявки
            start_device_service(device_id, request_id);
        }
    }
}
\end{lstlisting}

% ---- Результаты экспериментов ----
\section{Результаты работы имитационной модели}
\label{sec:results}
\subsection{Таблицы выходных характеристик}
\textbf{Пример результатов для базовой конфигурации:} $n=3$, $\tau=[3.0, 4.0, 5.0]$, $m=3$, $\mu=0.3$, $K=3$, $N=1000$.

\begin{table}[H]
\centering
\caption{Характеристики источников (Таблица 1)}
\label{tab:sources}
\begin{tabular}{@{}lrrrrrrr@{}}
\toprule
№ источника & \#заявок & $p_{отк}$ & $T_{преб}$ & $T_{БП}$ & $T_{обсл}$ & $Д_{БП}$ & $Д_{обсл}$ \\
\midrule
И1 & — & — & — & — & — & — & — \\
И2 & — & — & — & — & — & — & — \\
И3 & — & — & — & — & — & — & — \\
\bottomrule
\end{tabular}
\end{table}

\textit{Примечание: заполните таблицу результатами моделирования}

\begin{table}[H]
\centering
\caption{Характеристики приборов (Таблица 2)}
\label{tab:devices}
\begin{tabular}{@{}lr@{}}
\toprule
№ прибора & Коэффициент использования $K_{исп}$ \\
\midrule
П1 & — \\
П2 & — \\
П3 & — \\
\bottomrule
\end{tabular}
\end{table}

\textit{Примечание: заполните таблицу результатами моделирования}

\subsection{Графики зависимости выходных характеристик}
Вставьте графики зависимости выходных характеристик от входных параметров:

\begin{itemize}
    \item \textbf{Загрузка системы $\rho$:} $\rho = \frac{\sum_{i=1}^{n} \lambda_i}{\sum_{j=1}^{m} \mu_j} = \frac{\lambda_1 + \lambda_2 + \lambda_3}{m \cdot \mu}$
    \item \textbf{Вероятность отказа $p_{отк}(\rho)$:} зависимость вероятности отказа от загрузки системы
    \item \textbf{Коэффициент использования $K_{исп}(\rho)$:} зависимость загрузки приборов от загрузки системы
    \item \textbf{Среднее время ожидания $T_{БП}(\rho)$:} зависимость времени пребывания в буфере от загрузки системы
\end{itemize}

\begin{figure}[H]
    \centering
    %\includegraphics[width=0.8\textwidth]{graph_p_ref.pdf}
    \caption{Зависимость вероятности отказа от загрузки системы}
    \label{fig:graph_pref}
    \textit{Примечание: вставьте график $p_{отк}(\rho)$}
\end{figure}

\begin{figure}[H]
    \centering
    %\includegraphics[width=0.8\textwidth]{graph_utilization.pdf}
    \caption{Зависимость коэффициента использования от загрузки системы}
    \label{fig:graph_util}
    \textit{Примечание: вставьте график $K_{исп}(\rho)$}
\end{figure}

% ---- Анализ результатов ----
\section{Анализ результатов и выбор оптимальной конфигурации}
\label{sec:analysis}
\textbf{Процесс фильтрации конфигураций:}

Для каждой конфигурации из пространства поиска проверяются ограничения:
\begin{align}
p_{отк} &\le 0.05 \\
T_{преб} &\le 200 \text{ мс} \\
K_{исп} &\ge 0.70
\end{align}

Конфигурации, не удовлетворяющие хотя бы одному ограничению, отбрасываются.

\textbf{Ранжирование по критерию стоимости:}

Оставшиеся конфигурации ранжируются по возрастанию стоимости $C_{системы}$. Оптимальная конфигурация — минимальная стоимость при выполнении всех ограничений.

\begin{table}[H]
\centering
\caption{ТОП-10 оптимальных конфигураций}
\label{tab:top_configs}
\begin{tabular}{@{}lrrrrrr@{}}
\toprule
№ & $n$ & $m$ & $\mu$ & $K$ & $C_{системы}$, руб. & $p_{отк}$ & $T_{преб}$, мс \\
\midrule
1 & — & — & — & — & — & — & — \\
2 & — & — & — & — & — & — & — \\
\ldots & & & & & & & \\
\bottomrule
\end{tabular}
\end{table}

\textit{Примечание: заполните таблицу результатами исследования}

\textbf{Анализ влияния параметров:}

\begin{itemize}
    \item \textbf{Увеличение числа приборов:} снижает вероятность отказа и время ожидания, увеличивает стоимость системы
    \item \textbf{Увеличение ёмкости буфера:} снижает вероятность отказа при переполнении, незначительно влияет на стоимость
    \item \textbf{Увеличение интенсивности обслуживания:} снижает время пребывания и вероятность отказа, значительно увеличивает стоимость (более производительные серверы)
    \item \textbf{Оптимальный баланс:} достигается при комбинации параметров, обеспечивающей выполнение ограничений при минимальной стоимости
\end{itemize}

% ---- Выводы и рекомендации ----
\section{Вывод}
В ходе выполнения курсовой работы была разработана и исследована имитационная модель системы массового обслуживания с заданными дисциплинами работы.

\textbf{Достигнутые цели:}
\begin{itemize}
    \item Реализована имитационная модель СМО методом особых событий с использованием календаря событий
    \item Реализованы все дисциплины работы системы: Д10З3 (первое свободное место), Д10О4 (выбивание последней), Д2П2 (выбор прибора по кольцу), Д2Б3 (выбор из буфера по кольцу)
    \item Реализован графический интерфейс с пошаговым и автоматическим режимами работы
    \item Реализовано отображение динамики работы системы (временные диаграммы, календарь событий) и результатов моделирования (сводные таблицы)
    \item Проведено исследование влияния входных параметров на выходные характеристики системы
\end{itemize}

\textbf{Оптимальные конфигурации:} в результате исследования найдены конфигурации системы, удовлетворяющие требованиям к вероятности отказа ($p_{отк} \le 0.05$), времени пребывания ($T_{преб} \le 200$ мс) и коэффициенту использования ($K_{исп} \ge 0.70$) при минимальной стоимости.

\textbf{Практические рекомендации:}
\begin{itemize}
    \item Для системы обработки изображений рекомендуется конфигурация с балансом между количеством приборов и их производительностью
    \item Увеличение ёмкости буфера эффективно для снижения вероятности отказа при относительно низкой стоимости
    \item Необходимо учитывать требования к реальному времени при выборе конфигурации
\end{itemize}

\textbf{Ограничения модели:}
\begin{itemize}
    \item Не реализован автоматический процесс определения числа реализаций $N$ по алгоритму «пристрелки»
    \item Модель не учитывает дополнительные случайности (сбои оборудования, вариативность производительности)
    \item Не учитывается влияние температуры и других факторов на производительность GPU-серверов
\end{itemize}

\textbf{Пути улучшения:}
\begin{itemize}
    \item Реализация автоматического определения $N$ с проверкой заданной точности
    \item Добавление учёта дополнительных случайностей в модели
    \item Расширение экономической модели для более точной оценки стоимости системы
    \item Реализация экспорта результатов в CSV/Excel для дальнейшего анализа
\end{itemize}

% ---- Приложения ----
\appendix
\section{Приложение A: Формулы и определения}
\subsection{Расчёт числа реализаций $N$}
\[
N = \frac{t_{\alpha}^2 (1-p)}{p \delta^2}
\]
где $t_{\alpha} = 1.643$ для $\alpha = 0.9$, $\delta = 0.1$ (10\% точность), $p$ — вероятность отказа.

\subsection{Расчёт вероятности отказа}
\[
p_{отк} = \frac{m}{n}
\]
где $m$ — количество отказов, $n$ — количество поступивших заявок.

\subsection{Расчёт среднего времени пребывания}
\[
T_{преб} = T_{БП} + T_{обсл}
\]
где $T_{БП}$ — среднее время пребывания в буфере, $T_{обсл}$ — среднее время обслуживания.

\subsection{Расчёт коэффициента использования}
\[
K_{исп} = \frac{T_{занятости}}{T_{общее}}
\]
где $T_{занятости}$ — суммарное время занятости прибора, $T_{общее}$ — общее время моделирования.

\subsection{Расчёт загрузки системы}
\[
\rho = \frac{\sum_{i=1}^{n} \lambda_i}{\sum_{j=1}^{m} \mu_j} = \frac{\lambda_1 + \lambda_2 + \ldots + \lambda_n}{m \cdot \mu}
\]
где $\lambda_i$ — интенсивность $i$-го источника, $\mu$ — интенсивность обслуживания прибора, $m$ — количество приборов.

\subsection{Расчёт дисперсии}
\[
D = \frac{1}{n} \sum_{i=1}^{n} x_i^2 - \left( \frac{1}{n} \sum_{i=1}^{n} x_i \right)^2 = \overline{x^2} - \overline{x}^2
\]
где $x_i$ — значения случайной величины, $n$ — количество значений.

\section{Приложение B: Бланк задания (заполненный)}
Вставьте скан или текст заполненного бланка задания с указанием варианта и всех параметров.

\section{Приложение C: Исходные коды}
Исходные коды программы доступны в репозитории проекта. Основные файлы:
\begin{itemize}
    \item \texttt{src/core/Simulator.cpp} — основной класс симулятора
    \item \texttt{src/core/Buffer.cpp} — реализация буфера
    \item \texttt{src/core/Metrics.cpp} — сбор статистики
    \item \texttt{src/gui/MainWindow.cpp} — главное окно приложения
\end{itemize}

% Примеры ключевых функций приведены в разделе \ref{sec:software}.

\section{Приложение D: Полные таблицы результатов (CSV/XLSX)}
Полные таблицы результатов экспериментов сохранены в файлах формата CSV/Excel. Колонки таблиц:
\begin{itemize}
    \item Конфигурация: $n$, $m$, $\mu$, $K$
    \item Входные параметры: интервалы генерации, интенсивности источников
    \item Выходные характеристики по источникам: количество заявок, $p_{отк}$, $T_{преб}$, $T_{БП}$, $T_{обсл}$, дисперсии
    \item Выходные характеристики по приборам: $K_{исп}$
    \item Экономические показатели: стоимость системы
\end{itemize}

\end{document}